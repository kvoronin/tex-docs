\documentclass[a4paper,12pt]{amsart}
\input epsf
\usepackage{epsfig}
\usepackage{amsmath}

\DeclareMathAlphabet\mathbold{OML}{cmm}{b}{it}
\textheight=8.75in
\textwidth=6in
\oddsidemargin=-0.1in
\evensidemargin=-0.1in
\vsize 8.2in
% Equation numbering
\numberwithin{equation}{section}
 
\newtheorem{theorem}{Theorem}[section]
\newtheorem{lemma}{Lemma}[section]
\newtheorem{remark}{Remark}[section]
\newtheorem{corollary}{Corollary}[section]
\newtheorem{definition}{Definition}[section]
\newtheorem{algorithm}{Algorithm}[section]
\newtheorem{assumption}{Assumption}[section]
\newtheorem{example}{Example}[section]
\newtheorem{proposition}{Proposition}[section]





\newcommand{\curl}{\operatorname{curl}}
\renewcommand{\div}{\operatorname{div}}
\newcommand{\Div}{\operatorname{Div}}
\newcommand{\trace}{\operatorname{tr}}
\newcommand{\diag}{\operatorname{diag}}


\def\b1{{\mathbf 1}}
\def\bv{{\mathbf v}}
\def\bu{{\mathbf u}}
\def\bw{{\mathbf w}}
\def\bb{{\mathbf b}}
\def\bc{{\mathbf c}}
\def\bff{{\mathbf f}}
\def\bg{{\mathbf g}}
\def\bh{{\mathbf h}}
\def\br{{\mathbf r}}
\def\bs{{\mathbf s}}
\def\bd{{\mathbf d}}
\def\be{{\mathbf e}}
\def\bp{{\mathbf p}}
\def\bq{{\mathbf q}}
\def\bx{{\mathbf x}}
\def\by{{\mathbf y}}
\def\bz{{\mathbf z}}
\def\bn{{\mathbf n}}
\def\bbf{{\mathbf f}}
\def\bB{{\mathbf B}}
\def\bM{{\mathbf M}}
\def\bV{{\mathbf V}}
\def\bU{{\mathbf U}}
\def\bY{{\mathbf Y}}
\def\bF{{\mathbf F}}
\def\bA{{\mathbf A}}
\def\bB{{\mathbf B}}
\def\bC{{\mathbf C}}
\def\bD{{\mathbf D}}
\def\bN{{\mathbf N}}
\def\bT{{\mathbf T}}
\def\bP{{\mathbf P}}
\def\bQ{{\mathbf Q}}
\def\bS{{\mathbf S}}
\def\bR{{\mathbf R}}
 
\def\bepsilon{{\boldsymbol \epsilon}}
\def\balpha{{\boldsymbol \alpha}}
\def\bdelta{{\boldsymbol \delta}}
\def\blambda{{\boldsymbol \lambda}}
\def\bmu{{\boldsymbol \mu}}
 
 
 
%-----------------------------------------------------------------
\renewcommand{\O}{{\mathcal O}}
\newcommand{\Q}{{\mathcal Q}}
\newcommand{\R}{{\mathcal R}}
\newcommand{\A}{{\mathcal A}}
\newcommand{\B}{{\mathcal B}}
\newcommand{\C}{{\mathcal C}}
\newcommand{\D}{{\mathcal D}}
\newcommand{\Sc}{{\mathcal S}}
\newcommand{\F}{{\mathcal F}}
\newcommand{\G}{{\mathcal G}}
\newcommand{\I}{{\mathcal I}}
\newcommand{\J}{{\mathcal J}}
\newcommand{\M}{{\mathcal M}}
\newcommand{\N}{{\mathcal N}}
\newcommand{\X}{{\mathcal X}}
\newcommand{\Y}{{\mathcal Y}}
\newcommand{\calY}{{\mathcal Y}}
\newcommand{\calS}{{\mathcal S}}
\renewcommand{\L}{{\mathcal L}}
\renewcommand{\P}{{\mathcal P}}
 
\newcommand{\vertiii}[1]{{\left\vert\kern-0.25ex\left\vert\kern-0.25ex\left\vert #1 
    \right\vert\kern-0.25ex\right\vert\kern-0.25ex\right\vert}}
 
\newcommand{\V}{\text{\bf V}}
\newcommand{\K}{{\mathcal K}}
\newcommand{\T}{{\mathcal T}}
\newcommand{\E}{{\mathcal E}}

\newcommand{\hatcalK}{\widehat{\mathcal K}}
\newcommand{\hatcalS}{\widehat{\mathcal S}}
\newcommand{\hatA}{\widehat{A}}
%% \newcommand{\grad}{\nabla}
%% \renewcommand{\div}{\text{div}}
%% \renewcommand{\curl}{\text{curl}}
%
%
\def\XVec#1{{\mathbf #1}}
\def\XNorm#1{\left\| #1 \right\|}                       % norm
\def\XIProd#1#2{\left\langle #1 ,~ #2 \right\rangle}    % inner product
 
\def\XM{\mu}
 
\def\XQ{Q}                     % interpolant
\def\Xq#1{\XVec{q}_{#1}}       % rows of Q (interpolation to point #1)
\def\Xu{\XVec{u}}              % unknown vector
\def\Xf{\XVec{f}}              % right-hand-side vector
\def\Xe{\XVec{e}}
\def\Xr{\XVec{r}}
\def\Xx{\XVec{x}}
\def\Xv{\XVec{v}}
\def\Xw{\XVec{w}}
\def\Xn{\XVec{n}}              % normal vector
\def\Xes{\Xe_s}
\def\Xec{\Xe_c}
\def\Xus{\Xu_s}
\def\Xuc{\Xu_c}
\def\Xvs{\Xv_s}
\def\Xvc{\Xv_c}
\def\bone{{\boldsymbol 1}}
\def\bphi{{\boldsymbol \varphi}}
\def\bpsi{{\boldsymbol \psi}}
\def\bPsi{{\boldsymbol \Psi}}
\def\btheta{{\boldsymbol \theta}}
\def\bchi{{\boldsymbol \chi}}
\def\boldeta{{\boldsymbol \eta}}
\def\bolddelta{{\boldsymbol \delta}}
\def\bsigma{{\boldsymbol \sigma}}
\def\btau{{\boldsymbol \tau }}
\def\bxi{{\boldsymbol \xi }}
\def\bdelta{{\boldsymbol \delta}}
 
\def\Nedelec{N\'ed\'elec\ }

\def\bPi{{\boldsymbol \Pi}}
\def\bPhi{{\boldsymbol \Phi}}

\newcommand{\ott}[1]{\bar{#1}}
 

\newcommand{\dt}{\partial_t} 
\newcommand{\om}{\Omega} 

\renewcommand{\arraystretch}{1.2}

\DeclareMathOperator*{\argmin}{argmin}
 
 
\title[AMR with the minimization solver, report] 
{Adaptive refinement with the minimization solver, numerical results, report}

\address{Portland State University}

\keywords{CFOSLS, space-time, adaptive mesh refinement, multilevel algorithms, multigrid}

\begin{document}
 
\begin{abstract}
Adaptive mesh refinement is studied for constrained first-order system least-squares (CFOSLS) formulations for three-dimensional space-time problems. The central idea of the considered framework is to minimize the least-squares functional under the divergence constraint which makes the method conservative. Following this idea, we use local functional values as a natural measure of the local errors.

Two test problems are considered, one is the Laplace equation in an L-shaped domain and the second is the transport equation in $H(\div)-L^2$ formulation.

Efficiency of the proposed refinement strategy is compared numerically to the uniform refinement. In addition to the traditional refinement strategy based on local element errors, a modified strategy which noticeably smoothes out the error distribution is implemented by exploiting supplementary face error indicators.

For solving the arising linear systems, an efficient multilevel algorithm is proposed which minimizes the eneregy functional over a hierarchy of mesh levels. The multilevel algorithm outperforms the simpler block-diagonal preconditioners in terms of iteration count significantly.

The results obtained show that ...
\end{abstract}
\maketitle

\section{Introduction}

\section{Problem statement}
Description of the setup in general terms, applicable to both Laplace an transport.

\section{Refinement strategy}
Description of the element error indicators, and a modification with the face error indicators.

Definition of threshold in MFEM:
$$
\mbox{threshold} = \max \left\lbrace \mbox{total error} \cdot \mbox{total error fraction} \cdot n_{el}^{-\frac{1}{p}}, \, \mbox{local error goal} \right\rbrace.
$$

Total error is defined by
$$
\mbox{total error}  = 
\left\{ 
\begin{array}{lc}
\left( \sum_i \left(\mbox{local error i}\right)^p \right)^{1/p}, & p < \infty \\
\max_i \mbox{local error i}, & p = \infty \\
\end{array}
\right.
$$

Default parameter values are:
\begin{itemize}
	\item $\mbox{total error fraction} = 0.5$;
	\item $p = \infty$;
	\item $\mbox{local error goal} = 0.0$.
\end{itemize}

\section{Linear solvers}

\subsection{Block-diagonal preconditioners}

\subsection{Multilevel minimization solver}

\section{Results}

\subsection{Refinement stats?}

\subsection{Comparison to the uniform refinement}

Laplace

Transport

Pictures

\subsection{Solver comparison}

Block-diagonal preconditioner

Multilevel solver


For example, for the first two tables, total error fraction for the adaptive mesh refinement was 0.95.
This means, that with default values for the rest of parameters,
a refinement criteria was
$$
\mbox{local error i} \geq 0.95 \max_j \mbox{local error j}
$$

For the L-shaped domain, this strategy leads to the following refinement statistics:

\begin{table}[h!]
\caption{AMR stats, L-shaped domain in $\mathbb{R}^3$, $H(\div)-H^1$ formulation for the Laplace equation}
\label{tab:amr_stats_lshape3D_HdivH1lapl}
\scalebox{.85}{
\begin{tabular}{|c||c|c|c|c|} \hline
\#dofs & $n_{\mbox{el, marked}}$ & $\mbox{marked el,} \%$ & $n_{\mbox{el, new}}$ & $\mbox{new el,} \%$ \\ \hline
63677  & 6    & 0.03  & 362  & 1.86 \\ \hline 
64861  & 1    & 0.005 & 85   & 0.43 \\ \hline
65139  & 6    & 0.03  & 694  & 3.49 \\ \hline
67381  & 2    & 0.01  & 88   & 0.43 \\ \hline
67678  & 2    & 0.01  & 114  & 0.55 \\ \hline
68053  & 2    & 0.01  & 174  & 0.84 \\ \hline
68626  & -    & -     & -    & -    \\ \hline
\end{tabular}}
%checked
\end{table}

\begin{table}[h!]
\caption{AMR stats, L-shaped domain in $\mathbb{R}^3$, $H(\div)-H^1$ formulation for the Laplace equation}
\label{tab:amr_stats_lshape3D_HdivH1lapl}
\scalebox{.85}{
\begin{tabular}{|c||c|c|c|c|c|c|} \hline
\#dofs & \#iter1 & \#iter2 & $\varepsilon_{\bsigma}$ & $\varepsilon_u$ & funct & funct for $\pi exsol$ \\ \hline
63677  & 1           & 1 & 0.073 & 0.008  & 0.00029 & 0.00034 \\ \hline 
64861  & 1+5         & 2 & 0.068 & 0.0069 & 0.00025 & 0.00028 \\ \hline 
65139  & 1+5+4       & 2 & 0.067 & 0.0068 & 0.00024 & 0.00027 \\ \hline
67381  & 1+5+4+7     & 3 & 0.062 & 0.0056 & 0.0002  & 0.00023 \\ \hline
67678  & 1+5+4+7+3   & 2 & 0.061 & 0.0054 & 0.00019 & 0.00022 \\ \hline
68053  & 1+5+4+7+3+5 & 2 & 0.060 & 0.0053 & 0.00019 & 0.00022 \\ \hline

\end{tabular}}
\end{table}

Notations: \#iter1 is for the setup with re-solving from coarsest to finest level each time, \#iter2 is for simple minimization solver at the finest level. The numbers are the numbers of V-cycles so that the first number (1) in column 2 (\#iter1) corresponds to the single iteration (solution of the global system for a correction) at the coarsest level.

Next, the same test but with total error fraction equal to 0.8.

\begin{table}[h!]
\caption{AMR stats, L-shaped domain in $\mathbb{R}^3$, $H(\div)-H^1$ formulation for the Laplace equation}
\label{tab:amr_stats_lshape3D_HdivH1lapl}
\scalebox{.85}{
\begin{tabular}{|c||c|c|c|c|} \hline
\#dofs & $n_{\mbox{el, marked}}$ & $\mbox{marked el,} \%$ & $n_{\mbox{el, new}}$ & $\mbox{new el,} \%$ \\ \hline
63677  & 9    & 0.05  & 623  & 3.2 \\ \hline 
65709  & 15   & 0.07  & 859  & 4.2 \\ \hline
68501  & 50   & 0.24  & 3290 & 15.7 \\ \hline
79092  & 42   & 0.17  & 2272 & 9.4 \\ \hline
86468  & 10   & 0.04  & 949  & 3.6  \\ \hline
89541  & 111  & 0.4   & 6584 & 23.9 \\ \hline
110743 & -    & -     & -    & -    \\ \hline
\end{tabular}}
%checked
\end{table}

\begin{table}[h!]
\caption{AMR stats, L-shaped domain in $\mathbb{R}^3$, $H(\div)-H^1$ formulation for the Laplace equation}
\label{tab:amr_stats_lshape3D_HdivH1lapl}
\scalebox{.85}{
\begin{tabular}{|c||c|c|c|c|c|} \hline
\#dofs & \#iter & $\varepsilon_{\bsigma}$ & $\varepsilon_u$ & funct & funct for $\pi exsol$ \\ \hline
63677  & 1           & 0.073 & 0.008  & 0.00029 & 0.00034 \\ \hline 
65709  & 1+6         & 0.066 & 0.0065 & 0.00023 & 0.00026 \\ \hline 
68501  & 1+6+7       & 0.059 & 0.0052 & 0.00019 & 0.00022 \\ \hline
79092  & 1+6+7+5     & 0.052 & 0.0036 & 0.00013 & 0.00015 \\ \hline
86468  & 1+6+7+5+8   & 0.048 & 0.0031 & 0.00010 & 0.00013 \\ \hline
89541  & 1+6+7+5+8+4 & 0.047 & 0.0031 & 0.00009 & 0.00012 \\ \hline
\end{tabular}}
\end{table}

\begin{table}[h!]
\caption{Uniform refinement, L-shaped domain in $\mathbb{R}^3$, $H(\div)-H^1$ formulation for the Laplace equation}
\label{tab:ur_stats_lshape3D_HdivH1lapl}
\scalebox{.85}{
\begin{tabular}{|c||c|c|c|} \hline
\#dofs & \#iter & $\varepsilon_{\bsigma}$ & $\varepsilon_u$  \\ \hline
63677  & 9  & 0.073 & 0.008  \\ \hline 
501049 & 11 & 0.047 & 0.0032 \\ \hline 
\end{tabular}}
\end{table}
Here a MG preconditioner was used, \#levels = 2 and 3.


\section{Possible things for the paper}

\begin{itemize}
	\item Minimization solver
	\item AMR scheme for the minimization solver, three approaches
	\item Refinement strategies including beta	
	\item Laplace equation in the L-shaped domain and transport in the cube (pictures, comparison with uniform refinement and comparison between different approaches)
\end{itemize}

What would be the main concept (idea) of the paper?

We can say "we suggest a minimization solver for AMR in the considered CFOSLS setting".

I'd like to say that we can efficiently reuse the previous iterations in terms of the iteration count, but the results don't show it.

\section{To-do list}
\begin{itemize}
	\item Write the draft for theoretical sections
	\item Numerical results: tables
	\item Numerical resuls: pictures
	\item Introduction
\end{itemize}

\section{Questions:}

\begin{itemize}
	\item How to include results by Paulina?
\end{itemize}


%%%%%%%%%%%%%%%%%%%%%%%%%%%%%%%%%%%%%%
%%%%%%%%%%%%%%%%%%%%%%%%%%%%%%%%%%%%%%
\begin{thebibliography}{99}
\expandafter\ifx\csname url\endcsname\relax
  \def\url#1{\texttt{#1}}\fi
\expandafter\ifx\csname urlprefix\endcsname\relax\def\urlprefix{URL }\fi

\bibitem[WH]{mixedfem_adapt}
B.I. Wohlmuth and R. H. W. Ronald H. W. Hoppe. A comparison of a posteriori error estimators for mixed finite element discretizations by Raviart-Thomas elements. MATH.COMP, 68:1347–1378, 1999.

\bibitem[BMM]{fosls_adapt}
Local error estimates and adaptive refinement for first-order system least squares (FOSLS), M. Berndt, T. Manteuffel, and S. McCormick, E.T.N.A. 6 (1998), pp. 35-43.

\bibitem[AMMJT]{fosls_adapt2}
Efficiency-based adaptive local refinement for first-order system least-squares formulations, J. Adler, T. Manteuffel, S. McCormick, J. Nolting, J. Ruge, and L. Tang, SIAM J. Sci. Comp. 33 (2011), pp. 1-24. 

\bibitem[NVV]{neumueller_vassilevski_villa}
Martin Neumueller, Panayot S. Vassilevski, and Umberto Villa,
``{\em Space-time CFOSLS methods with AMGe upscaling,}''
Available as Lawrence Livermore National Laboratory Technical Report LLNL-CONF-683318, February 18, 2016.


\bibitem[V08]{MLBFP}
{\sc P.~S. Vassilevski,}
``{\em Multilevel Block--Factorization Preconditioners.} 
Matrix-based Analysis and Algorithms for Solving Finite Element Equations'',
Springer, New York, 2008.

\end{thebibliography}

\end{document}
